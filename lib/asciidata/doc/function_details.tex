\section{The detailed description}
\label{details}
The \AAD module was developed in Python using an Object Oriented (OO) approach
with classes and methods. This can not be hidden in the usage of the
\AAD module.
Working with \AAD means creating its class objects, accessing the class data
and executing class methods. This might be confusing for users who are
not familiar with this terminology and its meaning.

However this manual makes no attempt to introduce the OO terminology,
and its complete understanding is not really necessary in order to use
the \AAD module . The user can simply stick to
a strictly {\it phenomenological} approach by looking at the examples
and transferring them to his/her own applications. Nevertheless the OO terms
are used to structure this section of the manual.

%
% Section for functions
%
\subsection{Functions}
\label{functions}
\index{functions}
The \AAD module contains the two functions {\tt open()} and {\tt create()}.
These function serve as a starting point for the work with ASCII tables,
since both return an \ad object by either opening  and loading
an existing ASCII file ({\tt open()}) or creating an empty \ad object
from scratch ({\tt create()}).

\subsubsection{open()}
\label{functions_open}
\index{open()}\index{functions!open()}
This function loads an existing ASCII table file. An \ad object is
created and the data stored in the ASCII table is transferred to the
AsciiData object. Various function parameters specify e.g. the
character used as a delimiter to separate adjacent column elements.

\prgrf{Usage}
open(filename, null=None, delimiter=None, comment\_char=None)

\prgrf{Parameters}
\begin{tabular}{lccl}
Name     & Type    & Default & Description \\ \hline
filename &{\it string}& - & the name of the ASCII\\
         &            &   & table file to be loaded\\
null     &{\it string}& ['*','NULL', 'Null', 'None']& the character/string\\
         &            &   & representing a null-entry\\
delimiter&{\it string}& `` `` & the delimiter separating\\
         &            &   & columns \\
comment\_char  &{\it string}& '\#' & the character/string\\
         &            &   & indicating a comment\\
\end{tabular}

\prgrf{Return}
- an \ad object

\prgrf{Examples}
\begin{enumerate}
\item Load the file 'example.txt' and print the result. The file 'example.txt
looks like:
\begin{small}
\begin{verbatim}
#
# Some objects in the GOODS field
#
unknown  189.2207323  62.2357983  26.87  0.32
 galaxy  189.1408929  62.2376331  24.97  0.15
   star  189.1409453  62.1696844  25.30  0.12
 galaxy  188.9014716  62.2037839  25.95  0.20
\end{verbatim}
\end{small}
The command sequence is:
\begin{small}
\begin{verbatim}
>>> example = asciidata.open('example.txt')
>>> print example
#
# Some objects in the GOODS field
#
unknown  189.2207323  62.2357983  26.87  0.32
 galaxy  189.1408929  62.2376331  24.97  0.15
   star  189.1409453  62.1696844  25.30  0.12
 galaxy  188.9014716  62.2037839  25.95  0.20
\end{verbatim}
\end{small}

\item Load the file 'example2.txt' and print the results. 'example2.txt':
\begin{small}
\begin{verbatim}
@
@ Some objects in the GOODS field
@
unknown $ 189.2207323 $ 62.2357983 $ 26.87 $ 0.32
 galaxy $      *      $ 62.2376331 $ 24.97 $ 0.15
   star $ 189.1409453 $ 62.1696844 $ 25.30 $  *
 *      $ 188.9014716 $     *      $ 25.95 $ 0.20
\end{verbatim}
\end{small}

Load and print:
\begin{small}
\begin{verbatim}
>>> example2 = asciidata.open('example2.txt', null='*', \
                              delimiter='$', comment_char='@')
>>> print example2
@
@ Some objects in the GOODS field
@
unknown  $  189.2207323 $  62.2357983 $  26.87 $  0.32
 galaxy  $            * $  62.2376331 $  24.97 $  0.15
   star  $  189.1409453 $  62.1696844 $  25.30 $     *
       * $  188.9014716 $           * $  25.95 $  0.20
\end{verbatim}
\end{small}

\end{enumerate}

\subsubsection{create()}
\label{functions_create}
\index{create()}\index{functions!create()}

This function creates an empty \ad object in the 'plain' format, which means
that the column information is {\bf not} part of the default output.
The dimension of the \ad object as well as the delimiter separating
the elements is specified as input.

\prgrf{Usage}
create(ncols, nrows, null=None, delimiter=None)

\prgrf{Parameters}
\begin{tabular}{lccl}
Name     & Type    & Default & Description \\ \hline
ncols    &{\it int}& - & number of columns to be created\\
nrows    &{\it int}& - & number of rows to be created\\
null     &{\it string}& 'Null' & the character/string representing a null-entry\\
delimiter&{\it string}& `` `` & the delimiter separating the columns \\
\end{tabular}

\prgrf{Return}
- an \ad object in the 'plain' format

\prgrf{Examples}
\begin{enumerate}
\item Create an \ad object with 3 columns and 2 rows, print the result:
\begin{verbatim}
>>> example3 = asciidata.create(3,2)
>>> print (example3)
      Null       Null       Null
      Null       Null       Null
\end{verbatim}

\item As in 1., but use a different delimiter and NULL value, print the result:
\begin{verbatim}
>>> example4 = asciidata.create(3,2,delimiter='|', null='<*>')
>>> print (example4)
       <*> |        <*> |        <*>
       <*> |        <*> |        <*>
\end{verbatim}
\end{enumerate}


\subsubsection{createSEx()}
\label{functions_createSEx}
\index{createSEx()}\index{functions!createSEx()}

\prgrf{Usage}
createSEx(ncols, nrows, null=None, delimiter=None)

\prgrf{Parameters}
\begin{tabular}{lccl}
Name     & Type    & Default & Description \\ \hline
ncols    &{\it int}& - & number of columns to be created\\
nrows    &{\it int}& - & number of rows to be created\\
null     &{\it string}& 'Null' & the character/string representing a null-entry\\
delimiter&{\it string}& `` `` & the delimiter separating the columns \\
\end{tabular}

\prgrf{Return}
- an \ad object in the SExtractor catalogue format

\prgrf{Examples}
\begin{enumerate}
\item Create an \ad object with 3 columns and 2 rows, print the result:
\begin{verbatim}
>>> example5 = asciidata.createSEx(3,2)
>>> print example5
# 1  column1
# 2  column2
# 3  column3
      Null       Null       Null
      Null       Null       Null
\end{verbatim}

\item As in 1., but use a different delimiter and NULL value, print the result:
\begin{verbatim}
>>> example6 = asciidata.createSEx(3,2,delimiter='|', null='<*>')
>>> print example6
# 1  column1
# 2  column2
# 3  column3
       <*>|       <*>|       <*>
       <*>|       <*>|       <*>
\end{verbatim}
\end{enumerate}

